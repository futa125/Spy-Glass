\chapter{Zaključak i budući rad}
		
\justify{Zadatak zadan ovim projektom dovoljno je složen kako bi se članovi tima s nikakvim do malim iskustvom upoznali s mnogim problemima organizacije 
posla i vremena, programiranja, dizajna sustava i komunikacije unutar tima te tako dobili dojam o pravim problemima koje je potrebno rješiti kod
 razvoja programske podrške za stvarne klijente. Unatoč ambicioznom početku, većina članova tima su inicijalno slabo bili upoznati s razvojem
  programske podrške za sustav Android što je dodalo dodatnu razinu kompleksnosti na zadatak. Zajedničkim radom i suradnjom, projekt je u
   predefiniranom roku uspješno završen i tim je zadovoljan rezultatom.\par

Nakon sastavljanja tima, članovi tima krenuli su se upoznavati sa Android sustavom
 i alatima za razvoj programske podrške kao što je Git, Android studio i programski
  jezik Kotlin i sl. Nakon shvaćenog opsega projekta, članovi tima podijelili su se
   na zadatke koji najbolje odgovaraju sposobnostima pojedinog člana. Voditelj tima pratio je napredak svakog člana te 
   uskakao u pomoć ukoliko je to bilo potrebno. Iskusniji članovi tima izradili su kostur aplikacije dok su se ostali članovi posvetili 
   proučavanju zadatka u detalje i izradi tehničke dokumentacije kako bi za početak implementacije funkcionalnosti tim imao dobru podlogu.\par

Prva funkcionalnost koja je razvijena je sustav za registraciju i prijavu korisnika. Budući da je korištena dobro poznata i dokumentirana usluga 
Firebase, ova faza implementacije prošla je bez većih poteškoća.\par

Sljedeća funkcionalnost koja je implementirana je pospremanje tokena za push obavijesti u bazu podataka koja nije u cijelosti dokumentirana od 
strane Firebase usluge. Posljedica je funkcionalni sustav koji nije ostvaren najefikasnije moguće i podložan problemima kod proširenja aplikacije.\par

Naš najveći izazov bio je implementirati dinamičku detekciju dokumenta. Uspjeli smo to izvesti korištenjem Google-ovog ML Kita s kojim unutar 
implementacije CameraX API-ja provjeravamo je li na ekranu zapravo dokument. Osim toga nismo imali značajnijih izazova te su sve tražene funkcionalnosti
implementirane.

Projekt bi bio završen ranije da smo imali više iskustva s radom u timu te iskustva s radom u Androidu. No s vremenom je tim postao sve bolji te
smo brže i kvalitetnije implementirali tražene značajke.

Rad na projektu pokazao se dovoljno izazovnim da svaki od članova tima steče nova znanja o tehnologijama, timskom radu i svojim sposobnostima, a
u isto vrijeme dovoljno jednostavan da se tim rastane sa zadovoljstvom i iskustvom koje će im u budućoj karijeri olakšati prilagodbu ozbiljnom zadatku.

}
		
		\eject 